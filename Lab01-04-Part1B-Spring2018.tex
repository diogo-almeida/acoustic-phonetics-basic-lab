\section{Part 1b: Finding and measuring acoustic correlates of stop consonants}

In the first part of this lab, you explored the spectrographic representation of different vowels of English. Remember that the driving force behind this was the idea that our "one--to--one mapping" hypothesis about speech perception was worth exploring. In other words, we were exploring the idea that there are pieces of acoustic information that are reliably and univocally associated with segmental representations ($=$vowels and consonants). We started with vowels because in a way they are the ``simplest'' case of speech sounds.

You should think about what the first part of the lab has taught you. As far as vowels are concerned, do you think  that the ``one--to--one'' mapping hypothesis is plausible? Do you think the consonant data we will analyze will be consistent or inconsistent with what we have seen for vowels so far?

By now you already know how to\begin{inparaenum}[(a)]~\item do your own recordings, \item produce a spectrogram, \item identify and extract formant information from a spectrogram, \item save your recordings into a sound file, \item plot data in \MSExcel{}\end{inparaenum}. Therefore, the instructions from now on are going to be less detailed.

Now we will proceed to the logical extension of our hypothesis testing, i.e., we will be looking for acoustic correlates of stop consonants in English.The stop consonants we will be looking at are [p b t d k g]. This is the full inventory of stop consonants in English. Stop consonants receive that name because they are basically a full obstruction of the air flow, produced by either the closure of lips ([p b]), the touch of the tip of the tongue against the alveolar ridge ([t d]) or the touch of the back of the tongue against the velum ([k g]). The place where the flow of air is interrupted is called the \emph{place of articulation} of the consonant.

\subsection{Acoustic correlates of place of articulation in stop consonants}

\subsubsection{Before you start}

As you have probably realized by now, formants can be very visible or hard to see, depending on the vowels and the quality of the recording. In this part of the lab, you will be looking at full syllables. As you are going to realize, the shape of the formants seems to be an important piece of information about consonants. However, you can only analyze their shape if you can see them, right? In some cases, this will require some squinting on your part, especially when F1 and F2 are very close to each other. Don't dispair if you feel like you can't see anything. Try a couple of different recordings to see if it improves. If it all seems hopeless, then come see me.

\subsubsection{First step}

First, download \href{http://www.ling.umd.edu/~diogo/courses/ling499a/thebog-thedot-thegod.wav}{this sound file} (a native English speaker uttering \emph{the bog, the dot, the god}\footnote{This is Bill Idsardi, our resident phonologist. The sound file was adapted from some of the files he used for his LING253 class at University of Delaware. I have the feeling that the vowel he uses in \emph{bog} is different than the one he uses in \emph{dot} and \emph{god} (his lab didn't have the same purpose as ours, so it's my fault if that is the case). What do you think? Anyways, we can ignore this for the purposes of our lab; we are not going to analyze his data, but yours.}, open it in \Praat{} (\softmenu{Read} $>$ \softmenu{Read from file\ldots}) and go to the spectrogram view. Can you spot any differences between the consonants? This spectrogram should give you an idea of what you will be looking at from now on. Even though your recordings are probably going to be of lower quality than this one, this spectrogram should be a good reference for you.

\subsubsection{Bab Dad Gag}

Record the following three syllables in the same run: \emph{bab dad gag}. Fully articulate each word in a clear neutral tone and don't start articulating the next one before you have completely stopped uttering the word. Allow a little space between each word. It's okay to go back and forth until you get a good triplet.

Once you have a recording that you like, open the \softmenu{Edit} window. It should look like figure~\ref{step1triplet}:

\begin{figure}[!tbp]
\caption{\Praat{} -- Spectrograms of \emph{Bab Dad Gag}}
\label{step1triplet}
	\begin{center}
		\includegraphics[width=0.8\textwidth]{triplet-badaga1}
	\end{center}
\end{figure}

Notice that there are a lot of silence between words. You should delete them. Select portions that are clearly silence (you can play the selection in case you feel like double checking) and cut them, as shown in figure~\ref{step2triplet}.

\begin{figure}[!tbp]
\caption{\Praat{} -- Spectrograms of \emph{Bab Dad Gag} -- cutting silences}
\label{step2triplet}
	\begin{center}
		\includegraphics[width=0.8\textwidth]{triplet-badaga-cut1}
	\end{center}
\end{figure}

Try cutting as much silence as possible and leaving only the syllables in the display, like in figure~\ref{step3triplet}. If at this point you realize that your recording is not that good, feel free to make a new one. It might take a couple of trials to get a reasonable sample.

\begin{figure}[!tbp]
\caption{\Praat{} -- Spectrograms of \emph{Bab Dad Gag} -- final display}
\label{step3triplet}
	\begin{center}
		\includegraphics[width=0.8\textwidth]{triplet-badaga-cut-final1}
	\end{center}
\end{figure}

\paragraph{Write--up:} Once you have the triplet optimally displayed in front of you (silences cut, etc), try to figure out what the differences between the consonants are. Toggle between having the formant tracks on and off. What looks different in each spectrogram? Does the pattern of formants change according to the consonants you uttered? What about the shape of individual formants? Try to characterize as best as you can the differences you see. Do you think you can, at least tentatively, characterize the different consonants? That is, do you think you can come up with an acoustic definition of what [b] [d] and [g] are?

Remember to save a picture of your display\footnote{If you do not know how to do that, google ``Print Screen'' or ``Screen capture'' together with your operating system name. In case you want to edit (cropping, for instance) the image you captured, you'll have to use an image editor; \href{http://www.irfanview.com/}{IrfanView for Windows} comes to mind as a good simple alternative, \href{http://www.gimp.org/}{GIMP} is a far more complete and complex alternative, and it is cross--platform.} and put it in your write up, so I can see what you saw. Once you have a tentative acoustic definition of each consonant, move on to the next comparison.

\subsubsection{Bee Dee G(u)ee}

Do the same thing you did for the triplet \emph{bab, dad, gag}, but now record the following: \emph{bee}, \emph{dee}, \emph{g(u)ee} (like \emph{key} but with a [g] instead of [k]; not like the interjection \emph{gee}!)

\paragraph{Write--up:} Compare the three syllables. Remeber to try the formant tracks on and off (sometimes they hinder more than they help). What are the differences between the syllables? Are they consistent with the ones you observed for the first triplet? What about the tentative acoustic definitions for the consonants you derived from the previous triplet, do they hold for this new series of syllables?

Remember to save a picture and put it on your write up. Once you have finished this comparison and written it up, move on to the next one.

\subsubsection{Boo Doo Goo}

Do the same thing you did for the previous two triplets, but now record the following: \emph{boo}, \emph{doo}, \emph{goo}.

\paragraph{Write--up:} Compare the three syllables. What are the differences between them? Are they consistent with the ones you observed for the first and/or second triplet? What about the tentative acoustic definition for the consonants you derived from the previous triplets, does it hold for this new series of syllables?

Remember to save a picture and put it on your write up. Once you have finished this comparison and written it up, move on to the next one.

\subsubsection{Dee Daa Doo}

Do the same thing you did for the previous three triplets, but now record the following: \emph{dee}, \emph{daa}, \emph{doo}.

\paragraph{Write--up:} Compare the three syllables. Now, instead of having three different consonants and trying to characterize the differences between them, we are looking at the same consonant; therefore you should try to characterize the similarities between the graphs. By now, you have seen spectrograms for different syllables starting with [d]. Have they been consistent so far? Pay close attention at the shape of the formants. What consequences do you think this has for our ``one--to--one mapping'' hypothesis?

Remember to save a picture and put it on your write up. Once you have finished this comparison and written it up, you can move to the final part of the lab.
